\section{About Me}


\subsection{Contact Information}

\begin{itemize}

    \item \textbf{University} - Indian Institute of Technology, Jodhpur
    
    \item \textbf{Email IDs} - \href{mailto:gdp.1807@gmail.com}{gdp.1807@gmail.com}, \href{mailto:singh.23@iitj.ac.in}{singh.23@iitj.ac.in}
    
    \item \textbf{Github} - \href{https://github.com/czgdp1807}{czgdp1807}
    
    \item \textbf{Gitlab} - \href{https://gitlab.com/czgdp18071}{czgdp18071}
    
    \item \textbf{Timezone} - IST (UTC + 5:30)

\end{itemize}

\subsection{Personal Background}

I am a fourth year undergraduate in the department of Computer Science and Engineering at Indian Institute of Technology, Jodhpur.

I have completed the following relevant courses in my academic curriculum in the past years: Compiler Design, Computer Organisation and Architecture, Theory of Computation, Software Engineering, Object Oriented Analysis and Design.

\subsection{Programming Background}

I use Ubuntu 18.04.5 LTS as my operating system and Visual Studio Code as my editor. I have also worked with Eclipse IDE and Atom. I have written several lines of code in Python 3.x, C++11 and Java. I have also tried HTML, CSS and JavaScript in my initial years of undergraduate studies. A list of my ongoing and completed projects is as follows,

\begin{itemize}

    \item \justifying{\textbf{Enhancement of Statistics Module} - As a part of Google Summer of Code, 2019 with SymPy, in this project, I worked on designing an API framework for Stochastic Processes, Random Matrices and Multivariate
Probability Distributions under the guidance of my mentor Francesco Bonazzi. I also developed and implemented recursive algorithms in Python 3.x for automatically solving
probability and expectation queries for Markov Processes. The link to the report can be found \href{https://summerofcode.withgoogle.com/archive/2019/projects/4697311945424896/}{here}.}

    \item \justifying{\textbf{Anomaly Detection Engine} - This project was assigned to me during my internship at Morgan Stanley. I was responsible for designing and implementing an anomaly detection engine. Specifically, I implemented statistical algorithms in Java to detect anomalies in millions of data points consisting of strings, numerics and timestamps fetched from Elasticsearch servers using REST APIs. It is worthy to mention here that the anomaly detection engine is critical to teams involved in algorithmic trading to avoid economic losses due to anomalous data.}
    
    \item \justifying{\textbf{PyDataStructs} - As the name suggests it is a python package for data structures and algorithms. I started working on this in 2019 and since that point of time this project has successfully received contributions from various open source programs such as KWoC, GSSoC.}
    
    \item \justifying{\textbf{BNN} - BNN is a simple, light weight deep learning framework with a specialisation aimed for Binarized Neural Networks. I started working on this in 2020 and right now, I am heading towards implementing APIs for Layers, Activations and Models for MLPs.}

\end{itemize}  

Details on the above and more projects of mine are available on my Github profile. In all of the above projects I have used, \texttt{git} as my version control system. In addition, I have been using \textbf{Travis CI} in all of my projects except the one done at my internship.

\subsection{Previous Contributions to LFortran}

Following is the list of Merge Requests that I have made in the past to the Gitlab repository of LFortran,

\subsubsection{Merged}

\begin{itemize}

    \item \href{https://gitlab.com/lfortran/lfortran/-/merge_requests/755}{!735: Integer kinds}
    
    \item \href{https://gitlab.com/lfortran/lfortran/-/merge_requests/721}{!721: Adds pointers to LFortran}
    
    \item \href{https://gitlab.com/lfortran/lfortran/-/merge_requests/703}{!703: Added select case construct}
    
    \item \href{https://gitlab.com/lfortran/lfortran/-/merge_requests/698}{!698: Added Extraction of Kind from Integer Parameter}
    
    \item \href{https://gitlab.com/lfortran/lfortran/-/merge_requests/693}{!693: Support for BinOp and AnyTypeToReal for different Real Kinds}
    
    \item \href{https://gitlab.com/lfortran/lfortran/-/merge_requests/679}{!679: Implementing Real Kind}
    
    \item \href{https://gitlab.com/lfortran/lfortran/-/merge_requests/675}{!675: Logical types in LFortran}
    
    \item \href{https://gitlab.com/lfortran/lfortran/-/merge_requests/672}{!672: BinOp (Sub, Div, Mul, Pow) for Complex}
    
    \item \href{https://gitlab.com/lfortran/lfortran/-/merge_requests/666}{!666: Implemented Generic Code for ImplicitCast}
    
    \item \href{https://gitlab.com/lfortran/lfortran/-/merge_requests/671}{!671: Minor fixes for !666}
    
    \item \href{https://gitlab.com/lfortran/lfortran/-/merge_requests/657}{!657: Added Complex(expr, expr) to AST}
    
    \item \href{https://gitlab.com/lfortran/lfortran/-/merge_requests/771}{!771: Kinds for llvm::Type::getFloatTy and llvm::Type::getIntxxTy}
    
    \item \href{https://gitlab.com/lfortran/lfortran/-/merge_requests/722}{!722: Make integration\_tests/arrays\_01 work}

\end{itemize}

\subsubsection{Open}

\begin{itemize}
    
    \item \href{https://gitlab.com/lfortran/lfortran/-/merge_requests/691}{!691: Draft: map-$>$unordered\_map in asr\_to\_llvm.cpp}

\end{itemize}